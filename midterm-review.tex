\documentclass[draft]{article}
\usepackage{midterm-review}
\begin{document}
\begin{titlepage} % Suppresses displaying the page number on the title page and the subsequent page counts as page 1
    \newcommand{\HRule}{\rule{\linewidth}{0.5mm}} % Defines a new command for horizontal lines, change thickness here

    \centering % Centre everything on the page

    %------------------------------------------------
    %	Headings
    %------------------------------------------------

    \textsc{\LARGE University of Houston}\\[1.5cm] % Main heading such as the name of your university/college

    \textsc{\Large Foundations of Security}\\[0.5cm] % Major heading such as course name

    \textsc{\large COSC 6347}\\[0.5cm] % Minor heading such as course title

    %------------------------------------------------
    %	Title
    %------------------------------------------------

    \HRule\\[0.4cm]

    {\huge\bfseries Midterm Review}\\[0.4cm] % Title of your document

    \HRule\\[1.5cm]

    %------------------------------------------------
    %	Author(s)
    %------------------------------------------------

    \begin{minipage}{0.4\textwidth}
        \begin{flushleft}
            \large
            \textit{Author}\\
            K.M. \textsc{Hourani} % Your name
        \end{flushleft}
    \end{minipage}
    ~
    \begin{minipage}{0.4\textwidth}
        \begin{flushright}
            \large
            \textit{Based on Notes By}\\
            Dr. Aron \textsc{Laszka} % Supervisor's name
        \end{flushright}
    \end{minipage}

    % If you don't want a supervisor, uncomment the two lines below and comment the code above
    %{\large\textit{Author}}\\
    %John \textsc{Smith} % Your name

    %------------------------------------------------
    %	Date
    %------------------------------------------------

    \vfill\vfill\vfill % Position the date 3/4 down the remaining page

    {\large\today} % Date, change the \today to a set date if you want to be precise

    %------------------------------------------------
    %	Logo
    %------------------------------------------------

    %\vfill\vfill
    %\includegraphics[width=0.2\textwidth]{placeholder.jpg}\\[1cm] % Include a department/university logo - this will require the graphicx package

    %----------------------------------------------------------------------------------------

    \vfill % Push the date up 1/4 of the remaining page

\end{titlepage}

\section{Introduction to Security}
\subsection{Objectives}
\begin{center}
    \begin{tabular}{lll}
                                        & \multicolumn{1}{c}{Term}  & \multicolumn{1}{c}{Definition}             \\\toprule
        \multirow{3}{*}{\acrshort{cia}} & \Glsname{confidentiality} & \glstext*{confidentiality}                 \\
                                        & \Glsname{integrity}       & \glstext*{integrity}                       \\
                                        & \Glsname{availability}    & \glstext*{availability}                    \\\midrule
                                        & \Glsname{non-repudiation} & \multirow{2}{*}{\glsdesc*{accountability}} \\
                                        & \Glsname{accountability}  &                                            \\\midrule
                                        & \Glsname{privacy}         & \glstext*{privacy}                         \\\bottomrule
    \end{tabular}
\end{center}

\subsection{Challenges}
\Gls{weakest link} -- \glsdesc*{weakest link}

Security is a process, not a product -- attackers continuously looking for new vulnerabilities, so systems must be regularly updated and continuously monitored.

Tension between security and
\begin{itemize}[nosep]
    \item usability
    \item functionality
    \item efficiency
    \item time-to-market
    \item development cost
\end{itemize}

Value of security often only perceived when there is a security failure

Can be measured by
\begin{itemize}[nosep]
    \item checking compliance
    \item pentesting
\end{itemize}

\section{Introduction to Cryptography}
\subsection{Attacker Modeling Principles}
Security is defined with respect to an \gls{attacker model} -- what the attacker
\begin{itemize}[nosep]
    \item can do
    \item knows
    \item wants to achieve
\end{itemize}

Generally better to overestimate the attacker's capabilities, knowledge, and determination. Safe to assume attacker knows
\begin{itemize}[nosep]
    \item algorithms
    \item system design
    \item implementation
    \item configuaration
\end{itemize}
but the attacker cannot know \emph{truly} random values.

\subsection{Security by Obscurity}
\Gls{security by obscurity} -- \glsdesc*{security by obscurity}

Generally rejected by security experts, researchers, standard bodies, i.e., everyone.

Obscurity can slow down, but not stop, an attack:
\begin{itemize}[nosep]
    \item if we thought of something, attacker might also
    \item attacker might try attack for many possible design/implementation choices
\end{itemize}
Can create false sense of security.
\subsection{Symmetric-Key Ciphers}
Sender and receiver share a secret key $k$
\begin{center}
    \begin{tikzpicture}[node distance=3cm]
        \node[label=below:{$p$}] (p) {plaintext};
        \node[draw,rectangle,label=below:{encryption}, right of=p] (e) {$E$};
        \node[label=below:{$c=E(k,p)$}, right of=e] (c) {ciphertext};
        \node[draw,rectangle,label=below:{decryption}, right of=c] (d) {$D$};
        \node[label=below:{$p=D(k,c)$}, right of=d] (pd) {plaintext};
        \node[label=below:{$k$}, above=1cm of c] (k) {secret key};

        \draw[gray,thick,->] (p) -- (e);
        \draw[gray,thick,->] (e) -- (c);
        \draw[gray,thick,->] (c) -- (d);
        \draw[gray,thick,->] (d) -- (pd);
        \draw[gray,thick,->] (k.west) -- (e.north);
        \draw[gray,thick,->] (k.east) -- (d.north);
    \end{tikzpicture}
\end{center}
Types of attacks:
\begin{center}
    \begin{tabular}{lll}
        \multicolumn{1}{c}{Acronym} & \multicolumn{1}{c}{Attack}              & \multicolumn{1}{c}{Description}     \\\toprule
        \acrshort*{coa}             & \glsshortname{ciphertext only attack}   & \glstext*{ciphertext only attack}   \\
        \acrshort*{kpa}             & \glsshortname{known plaintext attack}   & \glstext*{known plaintext attack}   \\
        \acrshort*{cca}             & \glsshortname{chosen ciphertext attack} & \glstext*{chosen ciphertext attack} \\
        \acrshort*{cpa}             & \glsshortname{chosen plaintext attack}  & \glstext*{chosen plaintext attack}  \\
        \acrshort*{cta}             & \glsshortname{chosen text attack}       & \glstext*{chosen text attack}       \\\midrule
                                    & \glsshortname{brute-force attack}       & \glstext*{brute-force attack}       \\
                                    & \glsshortname{cryptanalytic attack}     & \glstext*{cryptanalytic attack}     \\\bottomrule
    \end{tabular}
\end{center}

\subsection{Kerckhoffs’s Principle}
\Glsname{Kerckhoffs} -- \glsdesc*{Kerckhoffs}. Rejection of \glsname{security by obscurity}
\section{Stream Ciphers}
\subsection{Perfect Security}
\Gls{perfect security} -- \glsdesc*{perfect security}

\Gls{one-time pad} -- \glsdesc*{one-time pad}

\subsection{Semantic Security}
\Gls{semantic security} -- \glsdesc*{semantic security}

Many-time pad: reusing the one-time key for multiple plaintext. Attacker can recover $p_1 \xor p_2$:
\begin{align*}c_1 \xor c_2
     & = (p_1 \xor k) \xor (p_2 \xor k) \\
     & = (p_1 \xor p_2) \xor (k \xor k) \\
     & = p_1 \xor p_2
\end{align*}
and if attacker knows $p_1$, can recover $p_2$:
\begin{align*}p_1 \xor (c_1 \xor c_2)
     & = p_1 \xor (p_1 \xor p_2) \\
     & = (p_1 \xor p_1) \xor p_2 \\
     & = p_2
\end{align*}
\subsection{General Model of Stream Ciphers}
Make one-time pad practical by securely extending the key.
\subsubsection*{Pseudorandom Number Generator}
\acrfull{prng} -- \glsdesc*{pseudorandom number generator}

Requirements:
\begin{itemize}[nosep]
    \item performance -- generates key as long as plaintext, so must be computationally efficient
    \item security -- generated sequence must be indistinguishable from true randomness
          \begin{itemize}
              \item \glsname{cryptanalytic attack}
                    \begin{itemize}[nosep]
                        \item uniform distribution -- 0s and 1s occur with approximately same frequency
                        \item independence -- no subsequence can be inferred from another, disjoint subsequence
                    \end{itemize}
              \item \glsname{brute-force attack}
                    \begin{itemize}[nosep]
                        \item $n$ bit key has $2^n$ possible values -- attacker can try all
                        \item key must be sufficiently long -- in 2014, NIST recommends 112-bits
                        \item as computers become faster, key length must be increased
                    \end{itemize}
          \end{itemize}
\end{itemize}

\subsubsection*{How Stream Cipher Works}
\glsname{stream cipher} -- \glsdesc*{stream cipher}

Use \acrshort{prng} to generate the sequence up to the length of the plaintext, then to
\begin{itemize}[nosep, align=left, leftmargin=1in]
    \item[\textbf{encrypt} ---] \texttt{xor} plaintext with key
    \item[\textbf{decrypt} ---] \texttt{xor} ciphertext with key
\end{itemize}
\subsection{Key-Reuse Problem}
If attacker learns $p_1 \xor p_2$, $p_2 \xor p_3$, $p_1 \xor p_3$, $\dots$, they can recover other plaintexts.
Solutions:
\begin{itemize}[nosep]
    \item one continuous sequence that allows seeking to any position in the key
    \item \glsname{nonce} -- \glsdesc{nonce}
          \begin{itemize}[nosep]\item xor key with \glsname{nonce} for each plaintext to produce different key\end{itemize}
\end{itemize}
\subsection{RC4}
Old WiFi and Web Security standard

\glsname{rc4}
Advantages
\begin{itemize}[nosep]
    \item variable key length (from 8 to 2048 bits)
    \item very simple, uses byte-oriented operations:
          \begin{itemize}[nosep]\item only 8 to 16 machine operations required per output byte\end{itemize}
\end{itemize}
Applications
\begin{itemize}[nosep]
    \item Wifi: WEP and WPA
          \begin{itemize}[nosep]\item broken in 2001, deprecated in 2004\end{itemize}
    \item Web Security (HTTPS): SSL and TLS
          \begin{itemize}[nosep]\item broken in 2013, deprecated in 2015\end{itemize}
\end{itemize}
RC4 has been retired.
\subsection{Salsa20/ChaCha20}
State of the Art Stream Cipher
\glsname{salsa20} (and more secure, more efficient variant \glsname{chacha20})

Key length is 128 or 256 bits.

\subsubsection*{Advantages}
\begin{itemize}[nosep]
    \item fast software implementation (simple 32-bit operations)
    \item can seek to any position in output sequence
    \item 64-bit nonce part of algorithm to prevent key-reuse
\end{itemize}
currently, no attacks better than \glsname{brute-force attack} known.

\subsubsection*{Algorithm}
\begin{itemize}[nosep]
    \item Output in blocks of $16\times32$ bits
    \item internal state: $16\times 32$ bits
          \begin{itemize}[nosep]\item initialized using key, nonce, and seek position\end{itemize}
    \item State updated with \texttt{xor}, 32-bit addition mod $2^{32}$, and rotating 32 bit values
    \item Performs 20 rounds of \texttt{xor}-add-rotate, each of which updates all values in state
    \item State added to original state to obtain output
\end{itemize}
\section{Block Ciphers}
Unlike \glspl{stream cipher}, \glspl{block cipher} have different encryption and decryption operations. A block cipher encrypts plaintext in fixed-length blocks
\begin{center}
    \begin{tikzpicture}[node distance=2.25cm, every node/.style={rectangle, draw, minimum width=2cm}]
        \node[label=above:{key}] (k) {$k$};
        \node[above right = 0.25cm and 0.5cm of k] (p1) {$p_1$};
        \node[right of=p1, label=above:{plaintext}] (p2) {$p_2$};
        \node[right of=p2] (p3) {$p_3$};

        \node[below of=p1] (e1) {$E(k, p_1)$};
        \node[below of=p2] (e2) {$E(k, p_2)$};
        \node[below of=p3] (e3) {$E(k, p_3)$};

        \node[below of=e1] (c1) {$c_1$};
        \node[below of=e2, label=below:{ciphertext}] (c2) {$c_2$};
        \node[below of=e3] (c3) {$c_3$};

        \draw[->] (p1) -- (e1);
        \draw[->] (e1) -- (c1);

        \draw[->] (p2) -- (e2);
        \draw[->] (e2) -- (c2);

        \draw[->] (p3) -- (e3);
        \draw[->] (e3) -- (c3);

        \draw[->] (k) -| ([xshift=0.5cm]e1.north west);
        \draw[->] (k) -| ([xshift=0.5cm]e2.north west);
        \draw[->] (k) -| ([xshift=0.5cm]e3.north west);
    \end{tikzpicture}
\end{center}
\subsection{Design Considerations}
\begin{itemize}[nosep]
    \item Key Size
          \begin{itemize}[nosep]
              \item number of possible $k$-bit keys is $2^k$
              \item $k$ must be sufficiently large to prevent \glsplural{brute-force attack}
          \end{itemize}
    \item Block Size
          \begin{itemize}
              \item too short $\rightarrow$ does not hide patterns in plaintext
                    \begin{itemize}
                        \item e.g. $n=8$ bits is 1 character
                        \item same as \glsname{substitution cipher}
                    \end{itemize}
              \item too long -- impractical, wasteful
          \end{itemize}
    \item encryption must be invertible
          \begin{itemize}
              \item different input blocks must be transformed into different output blocks
              \item can be viewed as a permutation on all $n$-bit blocks
              \item $(2^n)!$ possible permutations
          \end{itemize}
\end{itemize}
\subsection{Secure Block Cipher}
An $n$-bit \glsname{block cipher} is secure (for a computationally bounded attacker) if it is indistinguishable from a random permutation of $n$-bit blocks.

\glsname{diffusion} -- \glsdesc{diffusion}
\hrule
\begin{center}
    \begin{tikzpicture}[every node/.style={draw, rectangle}, minimum size=0.5cm, node distance=0.05cm]
        \node[label=left:{plaintext block}] (b1) {1};
        \node[right=of b1] (b2) {0};
        \node[right=of b2] (b3) {0};
        \node[right=of b3] (b4) {0};
        \node[right=of b4,draw=red!32!black,fill=yellow,thick] (b5) {1};
        \node[right=of b5] (b6) {0};
        \node[right=of b6] (b7) {0};
        \node[right=of b7] (b8) {1};
        \node[right=of b8] (b9) {1};
        \node[right=of b9] (b10) {1};
        \node[right=of b10] (b11) {0};
        \node[right=of b11] (b12) {1};

        \node[right=1cm of b12] (k1) {1};
        \node[right=of k1] (k2) {1};
        \node[right=of k2] (k3) {0};
        \node[right=of k3] (k4) {0};
        \node[right=of k4] (k5) {0};
        \node[right=of k5,label=right:{key}] (k6) {0};


        \node[below=2cm of b1, label=left:{ciphertext block}] (c1) {0};
        \node[below=2cm of b2,draw=red!32!black,fill=yellow,thick] (c2) {1};
        \node[below=2cm of b3,draw=red!32!black,fill=yellow,thick] (c3) {0};
        \node[below=2cm of b4] (c4) {1};
        \node[below=2cm of b5,draw=red!32!black,fill=yellow,thick] (c5) {0};
        \node[below=2cm of b6] (c6) {0};
        \node[below=2cm of b7] (c7) {1};
        \node[below=2cm of b8,draw=red!32!black,fill=yellow,thick] (c8) {1};
        \node[below=2cm of b9,draw=red!32!black,fill=yellow,thick] (c9) {0};
        \node[below=2cm of b10] (c10) {0};
        \node[below=2cm of b11] (c11) {1};
        \node[below=2cm of b12,draw=red!32!black,fill=yellow,thick] (c12) {0};

        \fill[gray!40] ([yshift=-0.25cm]b1.south west) rectangle ([yshift=0.25cm]c12.north east);
        \node[below=0.75cm of b1,draw=none,label=left:{encryption}] {};

        \draw[draw=none] (b5.south) -- ([yshift=-1.5cm]b5.south) node[draw=none, pos=0.5] (center) {};
        \path[draw,thick,red!32!black] (b5.south) -- (center.center);
        \path[draw,->,thick,red!32!black] (center.center) -| (c2);
        \path[draw,->,thick,red!32!black] (center.center) -| (c3);
        \path[draw,->,thick,red!32!black] (center.center) -| (c5);
        \path[draw,->,thick,red!32!black] (center.center) -| (c8);
        \path[draw,->,thick,red!32!black] (center.center) -| (c9);
        \path[draw,->,thick,red!32!black] (center.center) -| (c12);
    \end{tikzpicture}
\end{center}
\glsname{confusion} -- \glsdesc{confusion}
\hrule
\begin{center}
    \begin{tikzpicture}[every node/.style={draw, rectangle}, minimum size=0.5cm, node distance=0.05cm]
        \node[label=left:{plaintext block}] (b1) {1};
        \node[right=of b1] (b2) {0};
        \node[right=of b2] (b3) {0};
        \node[right=of b3] (b4) {0};
        \node[right=of b4] (b5) {1};
        \node[right=of b5] (b6) {0};
        \node[right=of b6] (b7) {0};
        \node[right=of b7] (b8) {1};
        \node[right=of b8] (b9) {1};
        \node[right=of b9] (b10) {1};
        \node[right=of b10] (b11) {0};
        \node[right=of b11] (b12) {1};

        \node[right=1cm of b12] (k1) {1};
        \node[right=of k1,draw=red!32!black,fill=yellow,thick] (k2) {1};
        \node[right=of k2] (k3) {0};
        \node[right=of k3] (k4) {0};
        \node[right=of k4] (k5) {0};
        \node[right=of k5,label=right:{key}] (k6) {0};

        \node[below=2cm of b1, label=left:{ciphertext block}] (c1) {0};
        \node[below=2cm of b2,draw=red!32!black,fill=yellow,thick] (c2) {1};
        \node[below=2cm of b3,draw=red!32!black,fill=yellow,thick] (c3) {0};
        \node[below=2cm of b4] (c4) {1};
        \node[below=2cm of b5,draw=red!32!black,fill=yellow,thick] (c5) {0};
        \node[below=2cm of b6] (c6) {0};
        \node[below=2cm of b7] (c7) {1};
        \node[below=2cm of b8,draw=red!32!black,fill=yellow,thick] (c8) {1};
        \node[below=2cm of b9,draw=red!32!black,fill=yellow,thick] (c9) {0};
        \node[below=2cm of b10] (c10) {0};
        \node[below=2cm of b11] (c11) {1};
        \node[below=2cm of b12,draw=red!32!black,fill=yellow,thick] (c12) {0};

        \fill[gray!40] ([yshift=-0.25cm]b1.south west) rectangle ([yshift=0.25cm]c12.north east);
        \node[below=0.75cm of b1,draw=none,label=left:{encryption}] {};

        \draw[draw=none] (k2.south) -- ([yshift=-1.5cm]k2.south) node[draw=none, pos=0.5] (center) {};
        \path[draw,thick,red!32!black] (k2.south) -- (center.center);
        \path[draw,->,thick,red!32!black] (center.center) -| (c2);
        \path[draw,->,thick,red!32!black] (center.center) -| (c3);
        \path[draw,->,thick,red!32!black] (center.center) -| (c5);
        \path[draw,->,thick,red!32!black] (center.center) -| (c8);
        \path[draw,->,thick,red!32!black] (center.center) -| (c9);
        \path[draw,->,thick,red!32!black] (center.center) -| (c12);
    \end{tikzpicture}
\end{center}

\subsection{Iterated Block Ciphers}
Hard to design a single invertible function that satisfies \glsname{diffusion} and \glsname{confusion}. Use a round function
\begin{itemize}[nosep]
    \item $R$ -- round function
          \begin{itemize}[nosep]
              \item relatively weak transformation that introduces \glsname{diffusion} and \glsname{confusion}
              \item by iterating, builds strong block cipher
          \end{itemize}
\end{itemize}
\begin{center}
    \begin{tikzpicture}[node distance=1cm]
        \node (k) {key $k$};
        \node[below left=2cm and 2.5cm of k] (k1) {$k_1$};
        \node[right= of k1] (k2) {$k_2$};
        \node[right= of k2] (k3) {$k_3$};
        \node[below right=2cm and 2.5cm of k] (km) {$k_m$};
        \draw[draw=none] (k3) -- (km) node[pos=0.5] (emptydots) {};


        \node[minimum size=1cm,draw,rectangle,below= of k1] (r1) {$R$};
        \node[minimum size=1cm,draw,rectangle,below= of k2] (r2) {$R$};
        \node[minimum size=1cm,draw,rectangle,below= of k3] (r3) {$R$};
        \node[below= 1.5cm of emptydots] (dots) {$\dots$};
        \node[minimum size=1cm,draw,rectangle,below= of km] (rm) {$R$};
        \node[left= of r1, align=center] (plaintext) {plaintext\\block};
        \node[right= of rm, align=center] (ciphertext) {ciphertext\\block};
        \node[left=of k1] {round keys};

        \fill[gray!40] (k1.north) -- ([yshift=-0.5cm]k.south west) -- ([yshift=-0.5cm]k.south east) -- (km.north);
        \node[below =0.75cm of k,align=center] {key\\expansion};

        \draw[->] (k) -- ([yshift=-0.5cm]k.south);

        \draw[->] (k1) -- (r1.north);
        \draw[->] (k2) -- (r2.north);
        \draw[->] (k3) -- (r3.north);
        \draw[->] (km) -- (rm.north);

        \draw[->] (plaintext) -- (r1);
        \draw[->] (r1) -- (r2);
        \draw[->] (r2) -- (r3);
        \draw[->] (r3) -- (dots);
        \draw[->] (dots) -- (rm);
        \draw[->] (rm) -- (ciphertext);
    \end{tikzpicture}
\end{center}

\subsection{Substitution-Permutation Ciphers}
Common subtype of iterated block cipher, each round $R$ consists of
\begin{itemize}[nosep]
    \item Substitution $S$
          \begin{itemize}[nosep]
              \item substitutes small block with another small block
              \item ideally, changing one input bit changes half of output bits
          \end{itemize}
    \item Permutation $P$
          \begin{itemize}[nosep]
              \item permutation of all bits
          \end{itemize}
\end{itemize}
\begin{center}
    \begin{tikzpicture}[every node/.style={rectangle, draw, minimum width=1.25cm}, node distance=0.25cm]
        \node[draw=none] (p) {plaintext block};
        \node[below left=0.5cm and 0.5cm of p] (firsts1) {$S_1$};
        \node[right=of firsts1] (firsts2) {$S_2$};
        \node[right=of firsts2] (firsts3) {$S_3$};
        \node[right=of firsts3] (firsts4) {$S_4$};
        \draw[draw=none] ([yshift=-1.5cm]firsts1.south west) rectangle ([yshift=-2cm]firsts4.south east) node[pos=0.5, minimum width=5.75cm] (mix1) {mix with key};
        \node[draw=none,right=of mix1] (k1) {$k_1$};
        \node[draw=none,below left=of firsts1] {$P$};
        \draw[->] ([xshift=0.25cm]k1.west) -- (mix1);

        \draw[draw=none] (mix1.north west) -- (mix1.north east)
        node[draw=none,pos=0.043] (firstmix1) {}
        node[draw=none,pos=0.087] (firstmix2) {}
        node[draw=none,pos=0.130] (firstmix3) {}
        node[draw=none,pos=0.174] (firstmix4) {}
        node[draw=none,pos=0.304] (firstmix5) {}
        node[draw=none,pos=0.348] (firstmix6) {}
        node[draw=none,pos=0.391] (firstmix7) {}
        node[draw=none,pos=0.435] (firstmix8) {}
        node[draw=none,pos=0.565] (firstmix9) {}
        node[draw=none,pos=0.609] (firstmix10) {}
        node[draw=none,pos=0.652] (firstmix11) {}
        node[draw=none,pos=0.696] (firstmix12) {}
        node[draw=none,pos=0.826] (firstmix13) {}
        node[draw=none,pos=0.870] (firstmix14) {}
        node[draw=none,pos=0.913] (firstmix15) {}
        node[draw=none,pos=0.957] (firstmix16) {};

        \draw[draw=none] (firsts1.south west) -- (firsts1.south east)
        node[draw=none, pos=0.2] (firsts11) {}
        node[draw=none, pos=0.4] (firsts12) {}
        node[draw=none, pos=0.6] (firsts13) {}
        node[draw=none, pos=0.8] (firsts14) {};
        \path[draw,->] (firsts11.center) -- ([yshift=-0.3cm]firsts11.south) -- ([yshift=0.3cm]firstmix6.north) -- (firstmix6.center);
        \path[draw,->] (firsts12.center) -- ([yshift=-0.3cm]firsts12.south) -- ([yshift=0.3cm]firstmix4.north) -- (firstmix4.center);
        \path[draw,->] (firsts13.center) -- ([yshift=-0.3cm]firsts13.south) -- ([yshift=0.3cm]firstmix9.north) -- (firstmix9.center);
        \path[draw,->] (firsts14.center) -- ([yshift=-0.3cm]firsts14.south) -- ([yshift=0.3cm]firstmix1.north) -- (firstmix1.center);

        \draw[draw=none] (firsts2.south west) -- (firsts2.south east)
        node[draw=none, pos=0.2] (firsts21) {}
        node[draw=none, pos=0.4] (firsts22) {}
        node[draw=none, pos=0.6] (firsts23) {}
        node[draw=none, pos=0.8] (firsts24) {};
        \path[draw,->] (firsts21.center) -- ([yshift=-0.3cm]firsts21.south) -- ([yshift=0.3cm]firstmix2.north) -- (firstmix2.center);
        \path[draw,->] (firsts22.center) -- ([yshift=-0.3cm]firsts22.south) -- ([yshift=0.3cm]firstmix7.north) -- (firstmix7.center);
        \path[draw,->] (firsts23.center) -- ([yshift=-0.3cm]firsts23.south) -- ([yshift=0.3cm]firstmix5.north) -- (firstmix5.center);
        \path[draw,->] (firsts24.center) -- ([yshift=-0.3cm]firsts24.south) -- ([yshift=0.3cm]firstmix14.north) -- (firstmix14.center);

        \draw[draw=none] (firsts3.south west) -- (firsts3.south east)
        node[draw=none, pos=0.2] (firsts31) {}
        node[draw=none, pos=0.4] (firsts32) {}
        node[draw=none, pos=0.6] (firsts33) {}
        node[draw=none, pos=0.8] (firsts34) {};
        \path[draw,->] (firsts31.center) -- ([yshift=-0.3cm]firsts31.south) -- ([yshift=0.3cm]firstmix11.north) -- (firstmix11.center);
        \path[draw,->] (firsts32.center) -- ([yshift=-0.3cm]firsts32.south) -- ([yshift=0.3cm]firstmix10.north) -- (firstmix10.center);
        \path[draw,->] (firsts33.center) -- ([yshift=-0.3cm]firsts33.south) -- ([yshift=0.3cm]firstmix16.north) -- (firstmix16.center);
        \path[draw,->] (firsts34.center) -- ([yshift=-0.3cm]firsts34.south) -- ([yshift=0.3cm]firstmix3.north) -- (firstmix3.center);

        \draw[draw=none] (firsts4.south west) -- (firsts4.south east)
        node[draw=none, pos=0.2] (firsts41) {}
        node[draw=none, pos=0.4] (firsts42) {}
        node[draw=none, pos=0.6] (firsts43) {}
        node[draw=none, pos=0.8] (firsts44) {};
        \path[draw,->] (firsts41.center) -- ([yshift=-0.3cm]firsts41.south) -- ([yshift=0.3cm]firstmix12.north) -- (firstmix12.center);
        \path[draw,->] (firsts42.center) -- ([yshift=-0.3cm]firsts42.south) -- ([yshift=0.3cm]firstmix8.north) -- (firstmix8.center);
        \path[draw,->] (firsts43.center) -- ([yshift=-0.3cm]firsts43.south) -- ([yshift=0.3cm]firstmix13.north) -- (firstmix13.center);
        \path[draw,->] (firsts44.center) -- ([yshift=-0.3cm]firsts44.south) -- ([yshift=0.3cm]firstmix15.north) -- (firstmix15.center);

        \node[below=2.5cm of firsts1] (seconds1) {$S_1$};
        \draw[draw=none] (seconds1.north west) -- (seconds1.north east)
        node[draw=none, pos=0.2] (top11) {}
        node[draw=none, pos=0.4] (top12) {}
        node[draw=none, pos=0.6] (top13) {}
        node[draw=none, pos=0.8] (top14) {};
        \node[right=of seconds1] (seconds2) {$S_2$};
        \draw[draw=none] (seconds2.north west) -- (seconds2.north east)
        node[draw=none, pos=0.2] (top21) {}
        node[draw=none, pos=0.4] (top22) {}
        node[draw=none, pos=0.6] (top23) {}
        node[draw=none, pos=0.8] (top24) {};
        \node[right=of seconds2] (seconds3) {$S_3$};
        \draw[draw=none] (seconds3.north west) -- (seconds3.north east)
        node[draw=none, pos=0.2] (top31) {}
        node[draw=none, pos=0.4] (top32) {}
        node[draw=none, pos=0.6] (top33) {}
        node[draw=none, pos=0.8] (top34) {};
        \node[right=of seconds3] (seconds4) {$S_4$};
        \draw[draw=none] (seconds4.north west) -- (seconds4.north east)
        node[draw=none, pos=0.2] (top41) {}
        node[draw=none, pos=0.4] (top42) {}
        node[draw=none, pos=0.6] (top43) {}
        node[draw=none, pos=0.8] (top44) {};
        \draw[->] ([yshift=0.35cm]top11.north) -- (top11.center);
        \draw[->] ([yshift=0.35cm]top12.north) -- (top12.center);
        \draw[->] ([yshift=0.35cm]top13.north) -- (top13.center);
        \draw[->] ([yshift=0.35cm]top14.north) -- (top14.center);
        \draw[->] ([yshift=0.35cm]top21.north) -- (top21.center);
        \draw[->] ([yshift=0.35cm]top22.north) -- (top22.center);
        \draw[->] ([yshift=0.35cm]top23.north) -- (top23.center);
        \draw[->] ([yshift=0.35cm]top24.north) -- (top24.center);
        \draw[->] ([yshift=0.35cm]top31.north) -- (top31.center);
        \draw[->] ([yshift=0.35cm]top32.north) -- (top32.center);
        \draw[->] ([yshift=0.35cm]top33.north) -- (top33.center);
        \draw[->] ([yshift=0.35cm]top34.north) -- (top34.center);
        \draw[->] ([yshift=0.35cm]top41.north) -- (top41.center);
        \draw[->] ([yshift=0.35cm]top42.north) -- (top42.center);
        \draw[->] ([yshift=0.35cm]top43.north) -- (top43.center);
        \draw[->] ([yshift=0.35cm]top44.north) -- (top44.center);

        \draw[draw=none] ([yshift=-1.5cm]seconds1.south west) rectangle ([yshift=-2cm]seconds4.south east) node[pos=0.5, minimum width=5.75cm] (mix2) {mix with key};
        \node[draw=none,right=of mix2] (k2) {$k_2$};
        \node[draw=none,below left=of seconds1] {$P$};
        \draw[->] ([xshift=0.25cm]k2.west) -- (mix2);

        \draw[draw=none] (mix2.north west) -- (mix2.north east)
        node[draw=none,pos=0.043] (secondmix1) {}
        node[draw=none,pos=0.087] (secondmix2) {}
        node[draw=none,pos=0.130] (secondmix3) {}
        node[draw=none,pos=0.174] (secondmix4) {}
        node[draw=none,pos=0.304] (secondmix5) {}
        node[draw=none,pos=0.348] (secondmix6) {}
        node[draw=none,pos=0.391] (secondmix7) {}
        node[draw=none,pos=0.435] (secondmix8) {}
        node[draw=none,pos=0.565] (secondmix9) {}
        node[draw=none,pos=0.609] (secondmix10) {}
        node[draw=none,pos=0.652] (secondmix11) {}
        node[draw=none,pos=0.696] (secondmix12) {}
        node[draw=none,pos=0.826] (secondmix13) {}
        node[draw=none,pos=0.870] (secondmix14) {}
        node[draw=none,pos=0.913] (secondmix15) {}
        node[draw=none,pos=0.957] (secondmix16) {};

        \draw[draw=none] (seconds1.south west) -- (seconds1.south east)
        node[draw=none, pos=0.2] (seconds11) {}
        node[draw=none, pos=0.4] (seconds12) {}
        node[draw=none, pos=0.6] (seconds13) {}
        node[draw=none, pos=0.8] (seconds14) {};
        \path[draw,->] (seconds11.center) -- ([yshift=-0.3cm]seconds11.south) -- ([yshift=0.3cm]secondmix6.north) -- (secondmix6.center);
        \path[draw,->] (seconds12.center) -- ([yshift=-0.3cm]seconds12.south) -- ([yshift=0.3cm]secondmix4.north) -- (secondmix4.center);
        \path[draw,->] (seconds13.center) -- ([yshift=-0.3cm]seconds13.south) -- ([yshift=0.3cm]secondmix9.north) -- (secondmix9.center);
        \path[draw,->] (seconds14.center) -- ([yshift=-0.3cm]seconds14.south) -- ([yshift=0.3cm]secondmix1.north) -- (secondmix1.center);

        \draw[draw=none] (seconds2.south west) -- (seconds2.south east)
        node[draw=none, pos=0.2] (seconds21) {}
        node[draw=none, pos=0.4] (seconds22) {}
        node[draw=none, pos=0.6] (seconds23) {}
        node[draw=none, pos=0.8] (seconds24) {};
        \path[draw,->] (seconds21.center) -- ([yshift=-0.3cm]seconds21.south) -- ([yshift=0.3cm]secondmix2.north) -- (secondmix2.center);
        \path[draw,->] (seconds22.center) -- ([yshift=-0.3cm]seconds22.south) -- ([yshift=0.3cm]secondmix7.north) -- (secondmix7.center);
        \path[draw,->] (seconds23.center) -- ([yshift=-0.3cm]seconds23.south) -- ([yshift=0.3cm]secondmix5.north) -- (secondmix5.center);
        \path[draw,->] (seconds24.center) -- ([yshift=-0.3cm]seconds24.south) -- ([yshift=0.3cm]secondmix14.north) -- (secondmix14.center);

        \draw[draw=none] (seconds3.south west) -- (seconds3.south east)
        node[draw=none, pos=0.2] (seconds31) {}
        node[draw=none, pos=0.4] (seconds32) {}
        node[draw=none, pos=0.6] (seconds33) {}
        node[draw=none, pos=0.8] (seconds34) {};
        \path[draw,->] (seconds31.center) -- ([yshift=-0.3cm]seconds31.south) -- ([yshift=0.3cm]secondmix11.north) -- (secondmix11.center);
        \path[draw,->] (seconds32.center) -- ([yshift=-0.3cm]seconds32.south) -- ([yshift=0.3cm]secondmix10.north) -- (secondmix10.center);
        \path[draw,->] (seconds33.center) -- ([yshift=-0.3cm]seconds33.south) -- ([yshift=0.3cm]secondmix16.north) -- (secondmix16.center);
        \path[draw,->] (seconds34.center) -- ([yshift=-0.3cm]seconds34.south) -- ([yshift=0.3cm]secondmix3.north) -- (secondmix3.center);

        \draw[draw=none] (seconds4.south west) -- (seconds4.south east)
        node[draw=none, pos=0.2] (seconds41) {}
        node[draw=none, pos=0.4] (seconds42) {}
        node[draw=none, pos=0.6] (seconds43) {}
        node[draw=none, pos=0.8] (seconds44) {};
        \path[draw,->] (seconds41.center) -- ([yshift=-0.3cm]seconds41.south) -- ([yshift=0.3cm]secondmix12.north) -- (secondmix12.center);
        \path[draw,->] (seconds42.center) -- ([yshift=-0.3cm]seconds42.south) -- ([yshift=0.3cm]secondmix8.north) -- (secondmix8.center);
        \path[draw,->] (seconds43.center) -- ([yshift=-0.3cm]seconds43.south) -- ([yshift=0.3cm]secondmix13.north) -- (secondmix13.center);
        \path[draw,->] (seconds44.center) -- ([yshift=-0.3cm]seconds44.south) -- ([yshift=0.3cm]secondmix15.north) -- (secondmix15.center);

        \node[draw=none,below=of mix2] {$\vdots$};
    \end{tikzpicture}
\end{center}
\subsection{DES}
\acrfull{des}
\begin{itemize}[nosep]
    \item block size -- 64 bits
    \item key size -- 56 bits
          \begin{itemize}[nosep]
              \item 56 bit random
              \item 8 bit parity check
          \end{itemize}
    \item iterated substitution cipher of 16 rounds
\end{itemize}

\begin{itemize}[nosep]
    \item initial permutation
          \begin{itemize}[nosep]
              \item no cryptographic significance
              \item facilities loading blocks in and out of 8-bit hardware
          \end{itemize}
    \item key permutation
          \begin{itemize}[nosep]
              \item discards parity bits
              \item no cryptographic significance
          \end{itemize}
\end{itemize}
\subsection{Feistel Network}
\section{Block Cipher Modes of Operation}
\section{Public-Key Encryption}
\section{Hash Functions}
\section{Message Authentication}
\section{Digital Signatuers}
\section{Key Distribution}
\section{Public-Key Distribution}
\clearpage

\glsaddall
\printglossary[nonumberlist]
\printglossary[nonumberlist,type=\acronymtype]

\end{document}
