\documentclass[draft]{article}
\usepackage{midterm-review}
\begin{document}
\begin{titlepage} % Suppresses displaying the page number on the title page and the subsequent page counts as page 1
    \newcommand{\HRule}{\rule{\linewidth}{0.5mm}} % Defines a new command for horizontal lines, change thickness here

    \centering % Centre everything on the page

    %------------------------------------------------
    %	Headings
    %------------------------------------------------

    \textsc{\LARGE University of Houston}\\[1.5cm] % Main heading such as the name of your university/college

    \textsc{\Large Foundations of Security}\\[0.5cm] % Major heading such as course name

    \textsc{\large COSC 6347}\\[0.5cm] % Minor heading such as course title

    %------------------------------------------------
    %	Title
    %------------------------------------------------

    \HRule\\[0.4cm]

    {\huge\bfseries Midterm Review}\\[0.4cm] % Title of your document

    \HRule\\[1.5cm]

    %------------------------------------------------
    %	Author(s)
    %------------------------------------------------

    \begin{minipage}{0.4\textwidth}
        \begin{flushleft}
            \large
            \textit{Author}\\
            K.M. \textsc{Hourani} % Your name
        \end{flushleft}
    \end{minipage}
    ~
    \begin{minipage}{0.4\textwidth}
        \begin{flushright}
            \large
            \textit{Based on Notes By}\\
            Dr. Aron \textsc{Laszka} % Supervisor's name
        \end{flushright}
    \end{minipage}

    % If you don't want a supervisor, uncomment the two lines below and comment the code above
    %{\large\textit{Author}}\\
    %John \textsc{Smith} % Your name

    %------------------------------------------------
    %	Date
    %------------------------------------------------

    \vfill\vfill\vfill % Position the date 3/4 down the remaining page

    {\large\today} % Date, change the \today to a set date if you want to be precise

    %------------------------------------------------
    %	Logo
    %------------------------------------------------

    %\vfill\vfill
    %\includegraphics[width=0.2\textwidth]{placeholder.jpg}\\[1cm] % Include a department/university logo - this will require the graphicx package

    %----------------------------------------------------------------------------------------

    \vfill % Push the date up 1/4 of the remaining page

\end{titlepage}

\section{Introduction to Security}
\subsection{Objectives}
\begin{center}
    \begin{tabular}{lll}
                                        & \multicolumn{1}{c}{Term}  & \multicolumn{1}{c}{Definition}             \\\toprule
        \multirow{3}{*}{\acrshort{cia}} & \Glsname{confidentiality} & \glstext*{confidentiality}                 \\
                                        & \Glsname{integrity}       & \glstext*{integrity}                       \\
                                        & \Glsname{availability}    & \glstext*{availability}                    \\\midrule
                                        & \Glsname{non-repudiation} & \multirow{2}{*}{\glsdesc*{accountability}} \\
                                        & \Glsname{accountability}  &                                            \\\midrule
                                        & \Glsname{privacy}         & \glstext*{privacy}                         \\\bottomrule
    \end{tabular}
\end{center}

\subsection{Challenges}
\Gls{weakest link} -- \glsdesc*{weakest link}

Security is a process, not a product -- attackers continuously looking for new vulnerabilities, so systems must be regularly updated and continuously monitored.

Tension between security and
\begin{itemize}[nosep]
    \item usability
    \item functionality
    \item efficiency
    \item time-to-market
    \item development cost
\end{itemize}

Value of security often only perceived when there is a security failure

Can be measured by
\begin{itemize}[nosep]
    \item checking compliance
    \item pentesting
\end{itemize}

\section{Introduction to Cryptography}
\subsection{Attacker Modeling Principles}
Security is defined with respect to an \gls{attacker model} -- what the attacker
\begin{itemize}[nosep]
    \item can do
    \item knows
    \item wants to achieve
\end{itemize}

Generally better to overestimate the attacker's capabilities, knowledge, and determination. Safe to assume attacker knows
\begin{itemize}[nosep]
    \item algorithms
    \item system design
    \item implementation
    \item configuaration
\end{itemize}
but the attacker cannot know \emph{truly} random values.

\subsection{Security by Obscurity}
\Gls{security by obscurity} -- \glsdesc*{security by obscurity}

Generally rejected by security experts, researchers, standard bodies, i.e., everyone.

Obscurity can slow down, but not stop, an attack:
\begin{itemize}[nosep]
    \item if we thought of something, attacker might also
    \item attacker might try attack for many possible design/implementation choices
\end{itemize}
Can create false sense of security.
\subsection{Symmetric-Key Ciphers}
Sender and receiver share a secret key $k$
\begin{center}
    \begin{tikzpicture}[node distance=3cm]
        \node[label=below:{$p$}] (p) {plaintext};
        \node[draw,rectangle,label=below:{encryption}, right of=p] (e) {$E$};
        \node[label=below:{$c=E(k,p)$}, right of=e] (c) {ciphertext};
        \node[draw,rectangle,label=below:{decryption}, right of=c] (d) {$D$};
        \node[label=below:{$p=D(k,c)$}, right of=d] (pd) {plaintext};
        \node[label=below:{$k$}, above=1cm of c] (k) {secret key};

        \draw[gray,thick,->] (p) -- (e);
        \draw[gray,thick,->] (e) -- (c);
        \draw[gray,thick,->] (c) -- (d);
        \draw[gray,thick,->] (d) -- (pd);
        \draw[gray,thick,->] (k.west) -- (e.north);
        \draw[gray,thick,->] (k.east) -- (d.north);
    \end{tikzpicture}
\end{center}
Types of attacks:
\begin{center}
    \begin{tabular}{lll}
        \multicolumn{1}{c}{Acronym} & \multicolumn{1}{c}{Attack}              & \multicolumn{1}{c}{Description}     \\\toprule
        \acrshort*{coa}             & \glsshortname{ciphertext only attack}   & \glstext*{ciphertext only attack}   \\
        \acrshort*{kpa}             & \glsshortname{known plaintext attack}   & \glstext*{known plaintext attack}   \\
        \acrshort*{cca}             & \glsshortname{chosen ciphertext attack} & \glstext*{chosen ciphertext attack} \\
        \acrshort*{cpa}             & \glsshortname{chosen plaintext attack}  & \glstext*{chosen plaintext attack}  \\
        \acrshort*{cta}             & \glsshortname{chosen text attack}       & \glstext*{chosen text attack}       \\\midrule
                                    & \glsshortname{brute-force attack}       & \glstext*{brute-force attack}       \\
                                    & \glsshortname{cryptanalytic attack}     & \glstext*{cryptanalytic attack}     \\\bottomrule
    \end{tabular}
\end{center}

\subsection{Kerckhoffs’s Principle}
\Glsname{Kerckhoffs} -- \glsdesc*{Kerckhoffs}. Rejection of \glsname{security by obscurity}
\section{Stream Ciphers}
\subsection{Perfect Security}
\Gls{perfect security} -- \glsdesc*{perfect security}

\Gls{one-time pad} -- \glsdesc*{one-time pad}

\subsection{Semantic Security}
\Gls{semantic security} -- \glsdesc*{semantic security}

Many-time pad: reusing the one-time key for multiple plaintext. Attacker can recover $p_1 \xor p_2$:
\begin{align*}c_1 \xor c_2
     & = (p_1 \xor k) \xor (p_2 \xor k) \\
     & = (p_1 \xor p_2) \xor (k \xor k) \\
     & = p_1 \xor p_2
\end{align*}
and if attacker knows $p_1$, can recover $p_2$:
\begin{align*}p_1 \xor (c_1 \xor c_2)
     & = p_1 \xor (p_1 \xor p_2) \\
     & = (p_1 \xor p_1) \xor p_2 \\
     & = p_2
\end{align*}
\subsection{General Model of Stream Ciphers}
Make one-time pad practical by securely extending the key.
\subsubsection*{Pseudorandom Number Generator}
\acrfull{prng} -- \glsdesc*{pseudorandom number generator}

Requirements:
\begin{itemize}[nosep]
    \item performance -- generates key as long as plaintext, so must be computationally efficient
    \item security -- generated sequence must be indistinguishable from true randomness
          \begin{itemize}
              \item \glsname{cryptanalytic attack}
                    \begin{itemize}[nosep]
                        \item uniform distribution -- 0s and 1s occur with approximately same frequency
                        \item independence -- no subsequence can be inferred from another, disjoint subsequence
                    \end{itemize}
              \item \glsname{brute-force attack}
                    \begin{itemize}[nosep]
                        \item $n$ bit key has $2^n$ possible values -- attacker can try all
                        \item key must be sufficiently long -- in 2014, NIST recommends 112-bits
                        \item as computers become faster, key length must be increased
                    \end{itemize}
          \end{itemize}
\end{itemize}

\subsubsection*{How Stream Cipher Works}
\glsname{stream cipher} -- \glsdesc*{stream cipher}

Use \acrshort{prng} to generate the sequence up to the length of the plaintext, then to
\begin{itemize}[nosep, align=left, leftmargin=1in]
    \item[\textbf{encrypt} ---] \texttt{xor} plaintext with key
    \item[\textbf{decrypt} ---] \texttt{xor} ciphertext with key
\end{itemize}
\subsection{Key-Reuse Problem}
If attacker learns $p_1 \xor p_2$, $p_2 \xor p_3$, $p_1 \xor p_3$, $\dots$, they can recover other plaintexts.
Solutions:
\begin{itemize}[nosep]
    \item one continuous sequence that allows seeking to any position in the key
    \item \glsname{nonce} -- \glsdesc{nonce}
          \begin{itemize}[nosep]\item xor key with \glsname{nonce} for each plaintext to produce different key\end{itemize}
\end{itemize}
\subsection{RC4}
Old WiFi and Web Security standard

\glsname{rc4}
Advantages
\begin{itemize}[nosep]
    \item variable key length (from 8 to 2048 bits)
    \item very simple, uses byte-oriented operations:
          \begin{itemize}[nosep]\item only 8 to 16 machine operations required per output byte\end{itemize}
\end{itemize}
Applications
\begin{itemize}[nosep]
    \item Wifi: WEP and WPA
          \begin{itemize}[nosep]\item broken in 2001, deprecated in 2004\end{itemize}
    \item Web Security (HTTPS): SSL and TLS
          \begin{itemize}[nosep]\item broken in 2013, deprecated in 2015\end{itemize}
\end{itemize}
RC4 has been retired.
\subsection{Salsa20/ChaCha20}
State of the Art Stream Cipher
\glsname{salsa20} (and more secure, more efficient variant \glsname{chacha20})

Key length is 128 or 256 bits.

\subsubsection*{Advantages}
\begin{itemize}[nosep]
    \item fast software implementation (simple 32-bit operations)
    \item can seek to any position in output sequence
    \item 64-bit nonce part of algorithm to prevent key-reuse
\end{itemize}
currently, no attacks better than \glsname{brute-force attack} known.

\subsubsection*{Algorithm}
\begin{itemize}[nosep]
    \item Output in blocks of $16\times32$ bits
    \item internal state: $16\times 32$ bits
          \begin{itemize}[nosep]\item initialized using key, nonce, and seek position\end{itemize}
    \item State updated with \texttt{xor}, 32-bit addition mod $2^{32}$, and rotating 32 bit values
    \item Performs 20 rounds of \texttt{xor}-add-rotate, each of which updates all values in state
    \item State added to original state to obtain output
\end{itemize}
\section{Block Ciphers}
\section{Block Cipher Modes of Operation}
\section{Public-Key Encryption}
\section{Hash Functions}
\section{Message Authentication}
\section{Digital Signatuers}
\section{Key Distribution}
\section{Public-Key Distribution}
\clearpage

\glsaddall
\printglossary[nonumberlist]
\printglossary[nonumberlist,type=\acronymtype]

\end{document}
