\documentclass{article}
\usepackage[utf8]{inputenc}
\usepackage{amsmath, amsthm, amssymb, enumitem, booktabs, mleftright, xfrac}
\UseCollection{xfrac}{plainmath}
\usepackage[colorlinks]{hyperref}
\usepackage[acronym, nopostdot]{glossaries}
\mleftright
\renewcommand{\Pr}{\operatorname{\mathbb{P}}}
\newcommand{\prob}[1]{\Pr\left(#1\right)}
\newcommand{\given}{\mid}
\makeglossaries
\loadglsentries{glossary.tex} 
\begin{document}
\section{Introduction to Security}
\subsection{Objectives}
\subsubsection{Confidentiality}
\subsubsection{Data Integrity}
\subsubsection{System Integrity}
\subsubsection{Availability}
\subsubsection{Non-Repudiation}
\subsubsection{Authenticity}
\subsubsection{Privacy}
\subsection{Attacker Modeling Principles}
\subsubsection{Safe Assumptions}
\subsubsection{Attacker Capabilities and Knowledge}
\subsubsection{Rejection of Security by Obscurity}

\section{Introduction to Cryptography}
\section{Stream Ciphers}
\section{Block Ciphers}
\section{Block Cipher Modes of Operation}
\section{Public-Key Encryption}
\section{Hash Functions}
\section{Message Authentication}
\section{Digital Signatuers}
\section{Key Distribution}
\section{Public-Key Distribution}
\clearpage

\glsaddall
\printglossary[nonumberlist]
\printglossary[nonumberlist,type=\acronymtype]

\end{document}
